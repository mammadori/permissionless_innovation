%%%%%%%%%%%%%%%%%%%%%%%%%%%%%%%%%%%%%%%%%%%%%%%%%%%%%%%%%%%%%%%%%%%%%%%%%%%%%
%
% Copyright 2016 - Marco Amadori <marco.amadori@gmail.com>
%
% Licenza: Creative Commons 4.0 CC-BY-SA
% 
% http://creativecommons.org/licenses/by-sa/4.0/it/deed.it
%
%%%%%%%%%%%%%%%%%%%%%%%%%%%%%%%%%%%%%%%%%%%%%%%%%%%%%%%%%%%%%%%%%%%%%%%%%%%%%

\documentclass[english,compress]{beamer}
\mode<presentation>{
  \usetheme{Warsaw}
  %\usecolortheme{wolverine}
  \useoutertheme[subsection=false]{smoothbars}
  \setbeamercovered{transparent}
  %\beamerdefaultoverlayspecification{<+->}
}

\usepackage[italian]{babel}
\usepackage[utf8]{inputenc}
\usepackage[T1]{fontenc}
\usepackage{times}

\usepackage{subfigure}
\usepackage{multicol}
\usepackage{amsmath}
\usepackage{epsfig}
\usepackage{graphicx}
\usepackage[all,knot]{xy}
\xyoption{arc}
\usepackage{url}
\usepackage{multimedia}
\usepackage{hyperref}
\usepackage{setspace}
\usepackage{textcomp}
\usepackage{soul}

\title[Bitcoin Overview]{Permissionless Innovation}
\subtitle{The Dawn of Decentralized Computing}
\author[Marco Amadori]{Marco <amadori@inbitcoin.it>}
\institute{inbitcoin for Speck \& Tech --- \url{https://inbitcoin.it}}
\date{\scriptsize Trento --- \vspace{.10cm}October, 24 - 2016}


%%%%% Figures %%%%%%
\pgfdeclareimage[height=0.5cm]{ccbysa}{figures/by-sa.pdf}
\pgfdeclareimage[height=1.6cm]{logo}{figures/inbitcoin_logo_1280.png}
\pgfdeclareimage[height=6cm]{vcmoney}{figures/vc_invest.png}
\pgfdeclareimage[height=6cm]{metrics}{figures/bitcoin_metrics2014.png}
\pgfdeclareimage[height=6cm]{overview}{figures/Bitcoin_Overview.png}
\pgfdeclareimage[height=6cm]{inflation}{figures/inflation.png}
\pgfdeclareimage[height=1.6cm]{bitcoin}{figures/Bitcoin.png}
\pgfdeclareimage[height=6cm]{whatis}{figures/WhatIsBitcoin.pdf}
\pgfdeclareimage[height=2cm]{challenge}{figures/Challenge.pdf}
\pgfdeclareimage[height=5cm]{transaction}{figures/Transaction_Double_Entry.png}
\pgfdeclareimage[height=3cm]{wallet}{figures/fbk_wallet.pdf}
\pgfdeclareimage[height=2cm]{keytoaddr}{figures/privk_to_pubK_to_addressA.png}
\pgfdeclareimage[height=6.5cm]{chain}{figures/ChainOfBlocks.png}
\pgfdeclareimage[height=3.3cm]{merkle}{figures/MerkleTree.png}
\pgfdeclareimage[height=6cm]{fullnode}{figures/fullnode.pdf}
\pgfdeclareimage[height=6cm]{hashrate}{figures/hashrate.png}
\pgfdeclareimage[height=6cm]{cryptocap}{figures/cryptomarketcap.png}
\pgfdeclareimage[height=4cm]{dao}{figures/1920px-Quadrant_Chart_for_Classifying_DAOs.png}
\pgfdeclareimage[height=3cm]{colored}{figures/colored-coin-diagram.jpg}

\newcommand{\powemph}[1]{\textbf{\underline{#1}}}
\newcommand{\challenge}{\pgfuseimage{challenge}}
\newcommand{\citazione}[2]{
  \begin{exampleblock}{}
  \begin{scriptsize}\emph{``{#1}''}\end{scriptsize}
  \vskip0.5mm
  \hspace*\fill{\begin{tiny}--- {#2} \end{tiny}}
  \end{exampleblock}
}

\begin{document}

\begin{frame}[plain]
  \titlepage
  \begin{center}%
    \pgfuseimage{bitcoin}%
    \hspace{1cm}%
    \pgfuseimage{logo}%
    \end{center}%
\end{frame}

\logo{\pgfuseimage{ccbysa}}

\section{What}

\begin{frame}[<+->]{Definitions}

\begin{exampleblock}{Definiton of Bitcoin}
Bitcoin is a peer-to-peer network
that maintains a public distributed ledger of 
digital math-based assets known as bitcoins.
\end{exampleblock}

\begin{exampleblock}{Definiton of Computer}
A computer is a system that takes data in input, elaborate them via a mathematical model and outputs data without interpreting them.
\end{exampleblock}

\onslide<+-> How hard is to forecast applications like videogames and social network out of this definition?


\end{frame}

\begin{frame}{What is ``Bitcoin''}{A catchy but misleading name}
\begin{itemize}
 \item Bitcoin is a two words name (Bit Coin)
 \item it is a nice name for a Company or a Product, not for a Protocol (FaceBook, WalMart)
 \item TCP/IP is not called Bitflux or Netwire (Trasmission Control Protocol/Internet Protocol)
 \item Bitcoin could better indentified as P2P/DCP (Peer-to-Peer Digital Currency Protocol)
\end{itemize}


\end{frame}

\begin{frame}{What is ``Bitcoin''}{The Currency, the Network, the Ledger}
\begin{center}
\pgfuseimage{whatis}
\end{center}
\end{frame}

\begin{frame}{The currency}{What is money?}
\begin{exampleblock}{Aristotle definition on money}
\begin{enumerate}
 \item It must be durable. Money must stand the test of time
 and the elements. It must not fade, corrode, or change through time.
 \item It must be portable. Money hold a high amount of 'worth' relative to its weight and size.
 \item It must be divisible. Money should be relatively easy to separate
 and re-combine without affecting its fundamental characteristics.
 \item It must have intrinsic value. This value of money should be independent 
 of any other object and contained in the money itself.
\end{enumerate}
\end{exampleblock}
\end{frame}

\begin{frame}{The currency}{What is money?}

\begin{exampleblock}{Modern definition on money}
\begin{itemize}
 \item Exchange of value
 \item Unit of account
 \item Store of value
\end{itemize}
\end{exampleblock}
\end{frame}




\begin{frame}{The Currency}{The Grey Metal Metaphore}

``As a thought experiment, imagine there was a base metal as scarce as gold but with the following properties:

\begin{itemize}
 \item boring grey in colour
 \item not a good conductor of electricity
 \item not particularly strong, but not ductile or easily malleable either
 \item not useful for any practical or ornamental purpose
 \item and one special, magical property: \powemph{can be transported over a communications channel}''
\end{itemize}

\em{Satoshi Nakamoto -- 27 August 2010}
\end{frame}

\begin{frame}{The Currency}{Some information}
 \begin{itemize}
  \item The supply of bitcoins is fixed at 21 millions, (now $\sim$16 M)
  \item Each bitcoin (BTC) can be divided in $10^8$ units (\mbox{1/100 000 000} is called one \emph{satoshi} )
  \item The network tends to produce 12.5 new bitcoins every 10 minutes (block reward)
  \item The block reward is halved every $\sim$4 years (210 000 blocks)
  \item We are in the $3^{nd}$ reward era out of 34 (rewards ends in 2140)
  \item 1 BTC = 600 € on online exchanges
 \end{itemize}

\end{frame}

\begin{frame}{The Currency}{Predictable Money Supply}
\pgfuseimage{inflation}
\end{frame}

\begin{frame}{Overview of the Network}
\pgfuseimage{overview}


\begin{scriptsize}Real time visuals: \url{http://bitcoinglobe.com/}\end{scriptsize}
\end{frame}

\section{Who}

\begin{frame}{Who Invented Bitcoin?}{Satoshi Nakamoto}

\begin{scriptsize}\begin{itemize}
\item Satoshi Nakamoto, in 2008 publishes a white paper, ``Bitcoin: a Peer-to-Peer Electronic Cash System'' via ``The Cryptography Mailing List''.
 \item In 2009--2011 he wrote a lot of posts (80000 words, the size of a novel) in flawless english with British colloquialisms (aside only the first post where he used American spellings).
 \item Satoshi is probably a pseudonym for a developer or a group, “vanished” from the web in April 2011 because he “moved to other things” 
 \item If he is not a group, he is a world class programmer, with deep knowledge of C++, economics, cryptography and peer-to-peer networking.
 \item His timestamps speculation are about either east-coast US with a fairly normal sleep schedule or western Europe with a \emph{coder} sleep schedule (probably not Japan)
\end{itemize}\end{scriptsize}

\challenge
\end{frame}

\section{Why}

\begin{frame}{Trusted third party}

\citazione{What is needed is an electronic payment system based on cryptographic proof
instead of trust, allowing any two willing parties to transact directly with each other 
without the need for a trusted third party.”}{Satoshi Nakamoto,
"Bitcoin: A Peer-to-Peer Electronic Cash System” -- October 31, 2008}

\begin{exampleblock}{}
Trustless does not mean that we do not need to trust \emph{anything}, but that we do not need to trust \emph{anyone}. 
\end{exampleblock}

\end{frame}

\section{How}

\begin{frame}{Consensus in a decentralized system}{The Byzantine Generals' problem}

\citazione{A group of generals of the Byzantine army camped with their troops around an 
enemy city. Communicating only by messenger, the generals must agree upon a 
common battle plan. However, one or more of them may be traitors who will try to 
confuse the others. The problem is to find an algorithm to ensure that the loyal 
generals will reach agreement.}{Marshall Pease, Robert Shosthak and Leslie Lamport, The Byzantine Generals Problem}

\end{frame}

\begin{frame}{Transactions}
  \pgfuseimage{transaction}
  
  \begin{minipage}{0.8 \textwidth}
   \citazione{\textbf{spending} is signing a transaction which transfers value from a previous transaction over to a new owner identified by a bitcoin address}{Andreas M. Antonopoulos -- Mastering Bitcoin -- O'Reilly 2014}
  \end{minipage}
\end{frame}

\begin{frame}{A Paper Wallet}{Vires in numeris}
 \pgfuseimage{wallet}
 \vfill
 \begin{tiny}
 \url{https://blockchain.info/address/1fbk5AYjA7wLdwbru2CunWEuToBu1USsX}
 \vfill
 \pgfuseimage{keytoaddr}
 \vfill
 Elliptic Curve Digital Signature Algorithm -- secp256k1 \\
 Hash = RIPEMD160(SHA256(pubkey))
\end{tiny}
\end{frame}

\begin{frame}{There are no transactions}{just scripts}
 \begin{itemize}
  \item A transaction (TX) is a script
  \begin{itemize}
    \item the script language is called ``Script'' language
    \item Script is a stack based language
    \item Not Turing Complete by choice
    \item A TX is valid if the script returns True
    \item A TX has \emph{k inputs} and \emph{j outputs}
   \end{itemize}
  
  \item Script P2PKH - Pay to Public Key Hash \\ 
      \texttt{OP\_DUP OP\_HASH160 <pubKeyHash> OP\_EQUALVERIFY OP\_CHECKSIG}
  \item Script P2SH - Pay to Script Hash \\ 
      \texttt{OP\_HASH160 <scriptHash> OP\_EQUAL}
 \end{itemize}
\end{frame}

\begin{frame}{Wallet Types}
 \begin{itemize}
  \item Hot Wallet (online wallet)
  \begin{itemize}
    \item Online Wallet without control of private keys	
    \item Online Wallet with control of private keys
    \item Desktop PC wallet (Full node, SPV Wallets)
    \item Smartphone Wallet
    \item Multisig multiplatform Wallet (Altana)
  \end{itemize}
  
  \item Cold Storage (disconnected wallet)
  \item Hardware Wallet (private keys not online)
  \item Brain Wallet (dangerous and powerful)
 \end{itemize}

\end{frame}


\section{The Blockchain}

\begin{frame}{Chain of Blocks}{A Distributed Ledger}
\begin{columns}
 \begin{column}{0.3 \textwidth}
  \pgfuseimage{chain}
 \end{column}
 \begin{column}{0.6 \textwidth}
 \begin{exampleblock}{\begin{scriptsize}Secure Hash Algorithm -- SHA256\end{scriptsize}}
  \begin{tiny}The proof of work used in Bitcoin takes advantage of the apparently random 
 nature of cryptographic hashes. A good cryptographic
 hash algorithm converts arbitrary data into a seemingly-random number.
 \end{tiny}
 \end{exampleblock}
  \pgfuseimage{merkle}
 \end{column}
 \hfill
\end{columns}
\end{frame}

\begin{frame}{Consensus via Proof of work}{Longest chain wins}
 \begin{columns}
  \begin{column}{0.5 \textwidth}
   \pgfuseimage{fullnode}
  \end{column}
  \begin{column}{0.4 \textwidth}
   \begin{exampleblock}{\begin{scriptsize}The ``Work'' is called ``mining'' \end{scriptsize}}
     \begin{scriptsize}
      \begin{enumerate}
       \item SHA256(SHA256(block header) + nonce) < \emph{target} ?
       \item The Bitcoin Network will reward me (25 BTC)
       \end{enumerate}
       \end{scriptsize}
\begin{tiny}
       This is a new type of Cryptographic Signature, a \textbf{DMMS} ---  \emph{Dynamic Membership Multi-party Signature}
     
     
       Difficulty (``inverse'' of target) will adapt to global hashrate
       every $\sim$ 2 weeks (2016 blocks)
      \end{tiny}


   \end{exampleblock}
   \vfill
  \end{column}
  \hfill
 \end{columns}
\end{frame}

\begin{frame}{Historycal Hashrate}{Logarithmic Scale}
 \pgfuseimage{hashrate}
\end{frame}

\section{Blockchain Apps}

% blockchain as Permanent DB 
\begin{frame}{Blockchain as DB}{Permanent Storage}
% Proof of existence
\begin{itemize}
 \item You could write important data in the Blockchain (for a small bitcoin fee)
 \item What is written in the Blockchain is ``forever''
 \item No one can remove or alter Blockchain information
 \item Example Application: Proof of Existence, Decentralization of Notary services 
\end{itemize}
 
\begin{tiny}
 \url{http://www.proofofexistence.com/detail/e3c21569e6ba5b488d5c416e8fc6ea166551cf64076f8f337ddc8cc8f9936bc0}
\end{tiny}
\end{frame}

\begin{frame}{Permissionless Innovation}{Bitcoin and Internet}
% Permissionless Innovation
\begin{itemize}
 \item Before Internet, point-to-point communication between computers was available
 \item You needed a contract or permission from a Telco in order to innovate
 \item Low level of Innovation, fax-machine, poor video conferences, not much more
 \item Bitcoin opens an era of financial Innovation (programmable money)
 \item The Blockchain permits Decentralized Computing
\end{itemize}
\end{frame}



\begin{frame}{Generic Asset Ledger}{Coloring Coins}
\begin{itemize}
\item Tracking bitcoin transaction to allow generic asset trading
\item ``coloring coins'' enables innovation on top
\item anyone can issue a colored coin
\end{itemize}
\vfill
\pgfuseimage{colored}
\end{frame}


\begin{frame}{Multisignatures}{Enabling Smart Contracts}
\begin{itemize}
\item Wallets that need more than one signature to send a transaction
\item k/n multisignatures are available in Bitcoin since 2012
\item Smart Contracts are Trustless Unbreakable Agreements
\item Example: micro and nanopayments trustless channels
\item Example: decentralized escrow (OpenBazaar is a decentralized Ebay)
\item Example: Smart Properties
\end{itemize}
\end{frame}

\begin{frame}{Scaling Bitcoin}{Need space?}
\begin{itemize}
\item a typical TX is 300 bytes, 1Mb each 10 minutes
\item \textbf{On chain TX are limited to 3 tps}
\item Paypal like scaling (Coinbase users)
\item Smaller TX (Segregated Witness, Schnorr signatures)
\item Blocksize increase (Hard Fork required)
\item Payment Channels
 \begin{itemize}
  \item already available
  \item exchange of presigned TX
  \item TX not published on blockchain, just Open/Close CH
 \end{itemize}
\end{itemize}
\end{frame} 

\begin{frame}{Scaling Bitcoin}{Need speed?}
\begin{itemize}
 \item Lightning Networks 
  \begin{itemize}
    \item Payment Channels on steroids with Routing
    \item Offchain Trustless Contracts > 1B tps
    \item Smart contract and Game Theory based
   \end{itemize}
 \item Sidechains
  \begin{itemize}
    \item an altchain with BTC as the currency
    \item bitcoin are locked in a chain to appear in the other chain
    \item chain can have different rules and features
    \item merged mining, federated mining
  \end{itemize}
\end{itemize}
\end{frame}

\begin{frame}{Security Issues}{Bitcoin attacks}
\begin{itemize}
 \item 51\% attack (50\% +1)
 \item Finney or “Block Withholding” Attack
 \item The Race Attack
 \item Key Guessing / Collision Attacks
 \item Non-Bitcoin / Infrastructure Attacks
 \item Non-Technical Attacks / Scams
\end{itemize}
\end{frame}


\section{Final Notes}

\begin{frame}{Why Cybersecutity and Bitcoin?}
\begin{itemize}
  \item \textbf{Money is Data, Data is Money}
  \item This dramatically increases the security requirements
  \item Keys must be stored safely
  \item New motivations (10 B\$ Bounty) for hacking of desktop and smarphone devices
  \item Security in a decentralized system
 \end{itemize}
\end{frame}


\begin{frame}{What Next?}{Questions?}
 \begin{itemize}
  \item Which topic needs more care?
  \item Security and Wallets handling
  \item Protocol security issues
  \item Cryptographic underliyng protocols security
  \item Trustless Smart Contracts
  \item New applications brainstorming
 \end{itemize}
 
  \begin{block}{These slides}
  \url{http://goo.gl/BbzhTT} [github: mammadori, branch ``speck'']
  \end{block}

\end{frame}


\end{document}

