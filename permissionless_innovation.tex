%%%%%%%%%%%%%%%%%%%%%%%%%%%%%%%%%%%%%%%%%%%%%%%%%%%%%%%%%%%%%%%%%%%%%%%%%%%%%
%
% Copyright 2014 - Marco Amadori <marco.amadori@gmail.com>
%
% Licenza: Creative Commons 3.0 CC-BY-SA
% 
% http://creativecommons.org/licenses/by-sa/3.0/it/deed.it
%
%%%%%%%%%%%%%%%%%%%%%%%%%%%%%%%%%%%%%%%%%%%%%%%%%%%%%%%%%%%%%%%%%%%%%%%%%%%%%

\documentclass[english,compress]{beamer}
\mode<presentation>{
  \usetheme{Warsaw}
  %\usecolortheme{wolverine}
  \useoutertheme[subsection=false]{smoothbars}
  \setbeamercovered{transparent}
  %\beamerdefaultoverlayspecification{<+->}
}

\usepackage[italian]{babel}
\usepackage[utf8]{inputenc}
\usepackage[T1]{fontenc}
\usepackage{times}

\usepackage{subfigure}
\usepackage{multicol}
\usepackage{amsmath}
\usepackage{epsfig}
\usepackage{graphicx}
\usepackage[all,knot]{xy}
\xyoption{arc}
\usepackage{url}
\usepackage{multimedia}
\usepackage{hyperref}
\usepackage{setspace}
\usepackage{textcomp}
\usepackage{soul}

\hypersetup{%
  colorlinks=false,% hyperlinks will be black
  linkbordercolor=red,% hyperlink borders will be red
  pdfborderstyle={/S/U/W 1}% border style will be underline of width 1pt
}


\title[Introduzione al Bitcoin]{Bitcoin, anche in pratica}
\subtitle{E un piccolo assaggio di altre novità}
\author[Marco Amadori]{\textcolor{white}{Marco Amadori <amadori@inbitcoin.it> \\ <marco.amadori@gmail.com>}}
\institute{\textcolor{white}{\textbf{\begin{large}InBitcoin.it                                                 \end{large}}}}
\date{\textcolor{white}{\scriptsize Riva del Garda --- \vspace{.10cm}5 Maggio 2015}}


%%%%% Figures %%%%%%
\pgfdeclareimage[height=0.5cm]{ccbysa}{figures/by-sa.pdf}
\pgfdeclareimage[width=\paperwidth,height=\paperheight]{trentino}{figures/trentino.jpg}
\pgfdeclareimage[height=6cm]{vcmoney}{figures/vc_invest.png}
\pgfdeclareimage[height=6cm]{metrics}{figures/bitcoin_metrics2014.png}
\pgfdeclareimage[height=6cm]{overview}{figures/Bitcoin_Overview.png}
\pgfdeclareimage[height=6cm]{inflation}{figures/inflation.png}
\pgfdeclareimage[height=1.6cm]{bitcoin}{figures/Bitcoin.png}
\pgfdeclareimage[height=1.4cm]{logouc}{figures/logo_cct.jpg}
\pgfdeclareimage[height=6cm]{whatis}{figures/WhatIsBitcoin.pdf}
\pgfdeclareimage[height=2cm]{challenge}{figures/Challenge.pdf}
\pgfdeclareimage[height=5cm]{transaction}{figures/Transaction_Double_Entry.png}
\pgfdeclareimage[height=3cm]{wallet}{figures/fbk_wallet.pdf}
\pgfdeclareimage[height=2cm]{keytoaddr}{figures/privk_to_pubK_to_addressA.png}
\pgfdeclareimage[height=6.5cm]{chain}{figures/ChainOfBlocks.png}
\pgfdeclareimage[height=3.3cm]{merkle}{figures/MerkleTree.png}
\pgfdeclareimage[height=6cm]{fullnode}{figures/fullnode.pdf}
\pgfdeclareimage[height=6cm]{hashrate}{figures/hashrate.png}
\pgfdeclareimage[height=6cm]{cryptocap}{figures/cryptomarketcap.png}
\pgfdeclareimage[height=4cm]{dao}{figures/1920px-Quadrant_Chart_for_Classifying_DAOs.png}
\pgfdeclareimage[height=3cm]{colored}{figures/colored-coin-diagram.jpg}

\newcommand{\powemph}[1]{\textbf{\underline{#1}}}
\newcommand{\challenge}{\pgfuseimage{challenge}}
\newcommand{\citazione}[2]{
  \begin{exampleblock}{}
  \begin{scriptsize}\emph{``{#1}''}\end{scriptsize}
  \vskip0.5mm
  \hspace*\fill{\begin{tiny}--- {#2} \end{tiny}}
  \end{exampleblock}
}

\begin{document}

\usebackgroundtemplate{\pgfuseimage{trentino}}

\begin{frame}[plain]
  \titlepage
  \begin{center}%
    \pgfuseimage{bitcoin}%
         \hspace{1cm}%
    \pgfuseimage{logouc}%

%     \url{http://inbitcoin.it}
    \end{center}%
\end{frame}

\logo{\pgfuseimage{ccbysa}}

\section{Cosa}

\usebackgroundtemplate{}

\begin{frame}{Che cosa è ``Bitcoin''}{La Valuta, La Rete, La Tecnologia, Il Database}
\begin{center}
\pgfuseimage{whatis}
\end{center}
\end{frame}

\begin{frame}{La Valuta}{Che cosa è ``Moneta''?}
\begin{exampleblock}{Definizioni Aristoteliche}
\begin{enumerate}
 \item Dev'essere duratura. La Moneta rimane inalterata nel tempo e agli elementi.
 \item Dev'essere portabile. Deve valere molto per quanto pesa e ingombrare poco.
 \item Dev'essere divisibile. Deve essere facile separare e riorganizzare la moneta
  senza modificarne le caratteristiche fondamentali.
 
 \item Deve avere valore intrinseco. Questo valore dev'essere indipendente da altri oggetti e contenuto nella moneta stessa.
 \end{enumerate}
\end{exampleblock}
\end{frame}

\begin{frame}{La valuta}{he cosa è ``moneta''?}

\begin{exampleblock}{Definizione più moderna}
\begin{itemize}
 \item Sistema di scambio di valore
 \item Unità di conto
 \item Riserva di valore (a breve, a lungo termine)
\end{itemize}
\end{exampleblock}
\end{frame}




\begin{frame}{La valuta}{Il Metallo Grigio}

\begin{small}``Immaginate un metallo simile all’oro quanto a scarsità di presenza
sulla superficie terrestre e quanto a difficoltà di estrazione e con le seguenti proprietà:

\begin{itemize}
 \item di colore grigiastro, per nulla attraente
 \item non duttile, non malleabile
 \item senza funzioni ornamentali o strutturali
 \item non un buon conduttore, né un buon isolante
 \smallskip
 \item però con una magica proprietà: \underline{\textbf{può essere trasmesso 
attraverso un canale di comunicazione}}.''
 \end{itemize}\end{small}
 \smallskip

\em{Satoshi Nakamoto -- 27 Agosto 2010}
\end{frame}

\begin{frame}{La Valuta}{Informazioni economiche}
 \begin{itemize}
  \item L'ammontare massimo di bitcoin è pari a 21 M, (ora siamo a $\sim$14 M)
  \item Esistono frazioni di bitcoin, può essere diviso in 100 milionesimi.
  \item La rete ``genera'' con il \emph{mining} in media 25 nuovi bitcoin ogni 10 minuti \begin{small}(Ricompensa del blocco o \emph{block reward})\end{small}
  \item La \emph{block reward} è automaticamente dimezzata ogni 4 anni. (210 000 blocchi)
  \item L'inflazione monetaria è dunque prevedibile con precisione
  \item 1 BTC = 210 € sui mercati (per auto-arbitraggio)
 \end{itemize}

\end{frame}

\begin{frame}{La valuta}{Inflazione monetaria}
\pgfuseimage{inflation}
\end{frame}

\begin{frame}{Sguardo dall'alto}
\pgfuseimage{overview}


\begin{scriptsize}Visualizzazione in tempo reale: \url{http://bitcoinglobe.com/}\end{scriptsize}
\end{frame}

\begin{frame}{Un portafoglio di carta}{Vis in numeris}
 \pgfuseimage{wallet}
 \vfill
 \begin{tiny}
 \url{https://blockchain.info/address/1fbk5AYjA7wLdwbru2CunWEuToBu1USsX}
 \vfill
 \pgfuseimage{keytoaddr}
 \vfill
 Crittografia a Curva Ellittica -- Sicurezza oltre il Militare, \href{http://miguelmoreno.net/wp-content/uploads/2013/05/fYFBsqp.jpg}{Termodinamica}
 \end{tiny}
\end{frame}

\section{Come}

\begin{frame}{Tipi di Portafoglio - Wallet Bitcoin}
   \begin{itemize}
    \item Online, Proprietario, modello ``bancario'', l'utente non ha il controllo delle chiavi segrete
    \item Online Wallet con controllo delle chiavi private o \underline{multifirma} \url{GreenAddress.it}
    \item Desktop Wallet (PC) --  Electrum, Multibit
    \item Smartphone Wallet -- Android Wallet, GreenAddress, Mycelium
    \item Multi Asset Wallet
     \item Cold Storage (wallet disconnessi)
    \item Hardware Wallet (le chiavi private non sono mai esposte)
  \end{itemize}

\end{frame}


\begin{frame}{Come si ottengono bitcoin}
   \begin{itemize}
    \item In cambio di beni e servizi, solitamente tramite Payment Processor (es: \href{https://youtu.be/GydkZrgv-YA}{Video POS NFC})
    \item Si possono comprare in cambio di Euro o Dollari
    \begin{itemize}
    
      \item Su mercati online, via bonifico, es: \href{https://www.kraken.com/charts}{kraken.com}, \href{https://www.bitstamp.net/market/tradeview/}{bitstamp.net}
      \item Via ricarica postepay o superflash alle poste o al tabacchino, es: \href{https://www.bitboat.net/it/buy}{bitboat.it}
      \item Da dei ``bancomat'' (Bitcoin ATM) come quello di Rovereto presso \href{https://twitter.com/mammadori/status/576505253564166145}{Ottica Guerra}, dove
      si possono comprare bitcoin in contanti.
    \end{itemize}
    \item Scambiandoli per altri ``crypto asset'' o altre monete matematiche
    \item {\st{``Minandoli'', con il proprio computer -- hardware specifico}}
  \end{itemize}

\end{frame}

\begin{frame}{Pagamenti in Bitcoin}{Come fare ad accettarli?}

N.B. Si \textbf{emette sempre lo scontrino in euro}, non (ancora) in bitcoin. Poi si sceglie tra 2 opzioni:

\begin{itemize}

 \item Usando un wallet direttamente, senza intermediari ma richiede know-how.
 \item Usando un App di un Payment Processor
\end{itemize}
\end{frame}

\begin{frame}{Pagamenti in Bitcoin}{Payment Processor}
 \begin{exampleblock}{Pagamenti Euro su Euro}
  \begin{enumerate}
   \item Il commerciante richiede Euro al cliente su un App (o in altro modo)
   
   \item Il Cliente paga ``tramite'' bitcoin dal cellulare \begin{small}(ma anche tablet, da un dispositivo
	 dedicato bitcoin o dal PC per il commercio online)\end{small}
 \item Il servizio di PP vende per lui i bitcoin sul mercato
 \item Il PP invia entro 1,2 giorni lavorativi un bonifico SEPA sul CC del commerciante
   ed ottiene Euro in conto corrente pari all'importo richiesto (anche 0\% commissioni)
  \end{enumerate}

 \end{exampleblock}

\end{frame}

\begin{frame}{Pagamenti via Bitcoin}{Svantaggi}

 \begin{itemize}
   \item Richiede (ancora) un accordo commerciale con l'azienda che fornisce il servizio, non Trustless
   \item I volumi sono ancora bassi
  \end{itemize}

\end{frame}

\begin{frame}{Pagamenti via Bitcoin}{Vantaggi}
 \begin{itemize}
  \item È \href{http://miguelmoreno.net/wp-content/uploads/2013/05/fYFBsqp.jpg}{sicuro} per design -- Push Vs Pull
  \item È \href{http://pagoinbit.it/pad/}{ubiquo}, disponibile ovunque
  \item Scherma il negoziante dalla volatilità del bitcoin come valuta
  \item Ha costi bassi o nulli per il commerciante
  \item Permette di ricevere pagamenti da una clientela globale
  \item Cattura per attuale scarsità di offerta il bitcoiner Olandese, Tedesco, Inglese..
  \item Marketing, Immagine Aziendale
  \item Permette Innovazione (mesh-network, nfc tags, micropagamenti, nanopagamenti, etc..)
  \end{itemize}

\end{frame}


\begin{frame}{Citazioni a caso}{Per prendere fiato}

\citazione{Il Bitcoin è una notevole conquista crittografica e la capacità di creare qualcosa che non è duplicabile 
nel mondo digitale ha un valore enorme.}{Eric Schmidt (ex CEO di Google)}

%\citazione{Every informed person needs to know about Bitcoin because it might be one of the world’s most important developments.}{Leon Louw, Nobel Peace prize nominee}

\citazione{Non aver avuto una strategia per Internet nel 1995 è equivalente a non avere una 
strategia per il Bitcoin ora.}{Moe Levin (CEO di Bitpay Europa)}

\citazione{Entro il 2005 sarà chiaro che l’impatto di Internet sull’economia non
sarà stato più grande di quello del fax.}{Paul Robin Krugman
-- Nobel Memorial Prize in Economic Sciences (1998)}
\end{frame}



\section{Extras}

\begin{frame}{Chain of Blocks}{A Distributed Ledger}
\begin{columns}
 \begin{column}{0.3 \textwidth}
  \pgfuseimage{chain}
 \end{column}
 \begin{column}{0.6 \textwidth}
 \begin{exampleblock}{\begin{scriptsize}Secure Hash Algorithm -- SHA256\end{scriptsize}}
  \begin{tiny}The proof of work used in Bitcoin takes advantage of the apparently random 
 nature of cryptographic hashes. A good cryptographic
 hash algorithm converts arbitrary data into a seemingly-random number.
 \end{tiny}
 \end{exampleblock}
  \pgfuseimage{merkle}
 \end{column}
 \hfill
\end{columns}
\end{frame}

\begin{frame}{Consensus via Proof of work}{Longest chain wins}
 \begin{columns}
  \begin{column}{0.5 \textwidth}
   \pgfuseimage{fullnode}
  \end{column}
  \begin{column}{0.4 \textwidth}
   \begin{exampleblock}{\begin{scriptsize}The ``Work'' is called ``mining'' \end{scriptsize}}
     \begin{scriptsize}
      \begin{enumerate}
       \item SHA256(SHA256(block header) + nonce) < \emph{target} ?
       \item The Bitcoin Network will reward me (25 BTC)
       \end{enumerate}
       \end{scriptsize}
\begin{tiny}
       This is a new type of Cryptographic Signature, a \textbf{DMMS} ---  \emph{Dynamic Membership Multi-party Signature}
     
     
       Difficulty (``inverse'' of target) will adapt to global hashrate
       every $\sim$ 2 weeks (2016 blocks)
      \end{tiny}


   \end{exampleblock}
   \vfill
  \end{column}
  \hfill
 \end{columns}
\end{frame}

\begin{frame}{Historycal Hashrate}{Logarithmic Scale}
 \pgfuseimage{hashrate}
\end{frame}



\section{Status}

\begin{frame}{Usage Metrics}{Latest quarter}
\pgfuseimage{metrics}
\begin{itemize}
 \item \url{http://coinmap.org/} 
\end{itemize}
\end{frame}

\begin{frame}{What is happening?}{Status of Venture Capitals}
  \pgfuseimage{vcmoney}
\end{frame}

\begin{frame}{Permissionless Innovation}{Bitcoin and Internet}
% Permissionless Innovation
\begin{itemize}
 \item Before Internet, point-to-point communication between computers was available
 \item You needed a contract or permission from a Telco in order to innovate
 \item Low level of Innovation, fax-machine, poor video conferences, not much more
 \item Bitcoin opens an era of financial Innovation (programmable money)
 \item The Blockchain permits Decentralized Computing
 \item Internet of Things: IBM's ``Adept'' will use the Blockchain 
\end{itemize}
\end{frame}

\section{Blockchain Apps}

% blockchain as Permanent DB 
\begin{frame}{Blockchain as DB}{Permanent Storage}
% Proof of existence
\begin{itemize}
 \item You could write important data in the Blockchain (for free or for a small fee)
 \item What is written in the Blockchain is ``forever''
 \item No one can remove or alter Blockchain information
 \item Example Application: Proof of Existence, Decentralization of Notary services 
\end{itemize}
 
\begin{tiny}
 \url{http://www.proofofexistence.com/detail/e3c21569e6ba5b488d5c416e8fc6ea166551cf64076f8f337ddc8cc8f9936bc0}
\end{tiny}
\end{frame}

\begin{frame}{Generic Asset Ledger}{Coloring Coins}
\begin{itemize}
\item Tracking bitcoin transaction to allow generic asset trading
\item ``coloring coins'' enables distributed exchanges
\item anyone can issue a colored coin
\end{itemize}
\vfill
\pgfuseimage{colored}
\end{frame}


\begin{frame}{Multisignatures}{Enabling Smart Contracts}
\begin{itemize}
\item Wallets that need more than one signature to send a transaction
\item k/n multisignatures are available in Bitcoin since 2012
\item Smart Contracts are Trustless Unbreakable Agreements
\item Example: micro and nanopayments trustless channels
\item Example: decentralized escrow (OpenBazaar is a decentralized Ebay)
\item Example: Smart Properties
\end{itemize}

\end{frame}


% colored coins, mastercoin, counterparty, sidechains
% multisignatures -> decentralized escrow, ebay, nanopayments channels
% smart contracts
% DAO ethereum, mastercoin, counterparty
% Bitcoin for good

\begin{frame}{Crypto currencies}{640 currencies should be enough for everyone}
 \begin{columns}
  \begin{column}{0.5 \textwidth}
 \pgfuseimage{cryptocap}  
  \end{column}
  
 \begin{column}{0.4 \textwidth}
\begin{itemize}
\begin{small}   
    \item Initially forks of bitcoin codebase
    \item Purpose-Specific or Experimental testbed
    \item Different parameters or hash Algorithm
    \item Due to being tied to bitcoin, they become real money too (crypto exchanges)
    \item Less network effect, no real threat to Bitcoin
\end{small}
   \end{itemize}
  \end{column}
  \hfill
  \end{columns}
\end{frame}

\begin{frame}{Appcoins}{Ethereum example}
\begin{itemize}
 
 \item Bitcoin full nodes execute a Non-Turing Complete script (handling of transactions, signatures)
 \item What if the script is Turing Complete?
 \item A platform for Smart Contracts \begin{scriptsize}\url{http://www.ethereum.org}\end{scriptsize}
 \item Distributed Applications
 \item Distributed Autonomous Corporations
 \end{itemize}

 \pgfuseimage{dao}
 
\end{frame}



\begin{frame}{Bitcoin for good}{Payment system for developing countries}
\begin{itemize}
 \item 50 \% of the world is unbanked
 \item Kenya: 40 \% of GDP is transacted via Mpesa, SMS money
 \item Remittances: $\sim$400 B\$ market, 8\% average fee
 \item Microcredits
\end{itemize}
\end{frame}

\section{Note Finali}


\begin{frame}{Domande}{Dubbi?}
 \begin{itemize}
 \item Sicurezza?
 \item Futuro della tecnologia?
 \item \textbf{Chiedete pure.}
  \end{itemize}
 
  \begin{block}{Queste slides}
  \url{http://goo.gl/BbzhTT} [github: mammadori, branch ``riva'']
  \end{block}
\begin{block}{Contatti}
  Marco Amadori <marco.amadori@gmail.com> \\
  \url{http://inbitcoin.it} --- \url{http://pagoinbit.it}
  \end{block}

\end{frame}


\end{document}

