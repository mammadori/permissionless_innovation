%%%%%%%%%%%%%%%%%%%%%%%%%%%%%%%%%%%%%%%%%%%%%%%%%%%%%%%%%%%%%%%%%%%%%%%%%%%%%
%
% Copyright 2014 - Marco Amadori <marco.amadori@gmail.com>
%
% Licenza: Creative Commons 3.0 CC-BY-SA
% 
% http://creativecommons.org/licenses/by-sa/3.0/it/deed.it
%
%%%%%%%%%%%%%%%%%%%%%%%%%%%%%%%%%%%%%%%%%%%%%%%%%%%%%%%%%%%%%%%%%%%%%%%%%%%%%

\documentclass[english,compress]{beamer}
\mode<presentation>{
  \usetheme{Warsaw}
  %\usecolortheme{wolverine}
  \useoutertheme[subsection=false]{smoothbars}
  \setbeamercovered{transparent}
  %\beamerdefaultoverlayspecification{<+->}
}

\usepackage[italian]{babel}
\usepackage[utf8]{inputenc}
\usepackage[T1]{fontenc}
\usepackage{times}

\usepackage{subfigure}
\usepackage{multicol}
\usepackage{amsmath}
\usepackage{epsfig}
\usepackage{graphicx}
\usepackage[all,knot]{xy}
\xyoption{arc}
\usepackage{url}
\usepackage{multimedia}
\usepackage{hyperref}
\usepackage{setspace}
\usepackage{textcomp}
\usepackage{soul}

\title[Bitcoin Research Challenges]{Permissionless Innovation}
\subtitle{Bitcoin Research Challenges \\ The Dawn of Decentralized Computing}
\author[Marco Amadori]{Marco Amadori <amadori@fbk.eu> \\ <marco.amadori@gmail.com>}
\institute{Fondazione Bruno Kessler --- \url{https://www.fbk.eu}}
\date{\scriptsize Trento --- \vspace{.10cm}10 October 2014}


%%%%% Figures %%%%%%
\pgfdeclareimage[height=0.5cm]{ccbysa}{figures/by-sa.pdf}
\pgfdeclareimage[height=1.6cm]{logo}{figures/fbk_logo.jpg}
\pgfdeclareimage[height=6cm]{vcmoney}{figures/vc_invest.png}
\pgfdeclareimage[height=6cm]{metrics}{figures/bitcoin_metrics2014.png}
\pgfdeclareimage[height=6cm]{overview}{figures/Bitcoin_Overview.png}
\pgfdeclareimage[height=6cm]{inflation}{figures/inflation.png}
\pgfdeclareimage[height=1.6cm]{bitcoin}{figures/Bitcoin.png}
\pgfdeclareimage[height=6cm]{whatis}{figures/WhatIsBitcoin.pdf}
\pgfdeclareimage[height=2cm]{challenge}{figures/Challenge.pdf}
\pgfdeclareimage[height=5cm]{transaction}{figures/Transaction_Double_Entry.png}
\pgfdeclareimage[height=3cm]{wallet}{figures/fbk_wallet.pdf}
\pgfdeclareimage[height=2cm]{keytoaddr}{figures/privk_to_pubK_to_addressA.png}
\pgfdeclareimage[height=6.5cm]{chain}{figures/ChainOfBlocks.png}
\pgfdeclareimage[height=3.3cm]{merkle}{figures/MerkleTree.png}
\pgfdeclareimage[height=6cm]{fullnode}{figures/fullnode.pdf}
\pgfdeclareimage[height=6cm]{hashrate}{figures/hashrate.png}
\pgfdeclareimage[height=6cm]{cryptocap}{figures/cryptomarketcap.png}



\newcommand{\powemph}[1]{\textbf{\underline{#1}}}
\newcommand{\challenge}{\pgfuseimage{challenge}}
\newcommand{\citazione}[2]{
  \begin{exampleblock}{}
  \begin{scriptsize}\emph{``{#1}''}\end{scriptsize}
  \vskip0.5mm
  \hspace*\fill{\begin{tiny}--- {#2} \end{tiny}}
  \end{exampleblock}
}

\begin{document}

\begin{frame}[plain]
  \titlepage
  \begin{center}%
    \pgfuseimage{bitcoin}%
    \hspace{1cm}%
    \pgfuseimage{logo}%
    \end{center}%
\end{frame}

\logo{\pgfuseimage{ccbysa}}

\section{What}

\begin{frame}[<+->]{Definitions}

\begin{exampleblock}{Definiton of Bitcoin}
Bitcoin is a peer-to-peer network
that maintains a public distributed ledger of 
digital math-based assets known as bitcoins.
\end{exampleblock}

\begin{exampleblock}{Definiton of Computer}
A computer is a system that takes data in input, elaborate them via a mathematical model and outputs data without interpreting them.
\end{exampleblock}

\onslide<+-> How hard is to forecast applications like videogames and social network out of this definition?


\end{frame}

\begin{frame}{What is ``Bitcoin''}{The Currency, the Network, the Ledger}
\begin{center}
\pgfuseimage{whatis}
\end{center}
\end{frame}

\begin{frame}{The Currency}{The Grey Metal Metaphore}

``As a thought experiment, imagine there was a base metal as scarce as gold but with the following properties:

\begin{itemize}
 \item boring grey in colour
 \item not a good conductor of electricity
 \item not particularly strong, but not ductile or easily malleable either
 \item not useful for any practical or ornamental purpose
 \item and one special, magical property: \powemph{can be transported over a communications channel}''
\end{itemize}

\em{Satoshi Nakamoto -- 27 August 2010}
\end{frame}

\begin{frame}{The Currency}{Some information}
 \begin{itemize}
  \item The supply of bitcoins is fixed at 21 millions, (now $\sim$13 M)
  \item Each bitcoin (BTC) can be divided in $10^8$ units (\mbox{1/100 000 000} is called one \emph{satoshi} )
  \item The network tends to produce 25 new bitcoins every 10 minutes (block reward)
  \item The block reward is halved every $\sim$4 years (210 000 blocks)
  \item We are in the $2^{nd}$ reward era out of 34 (rewards ends in 2140)
  \item 1 BTC = 310 € on online exchanges
 \end{itemize}

\end{frame}

\begin{frame}{The Currency}{Predictable Money Supply}
\pgfuseimage{inflation}
\end{frame}

\begin{frame}{Overview of the Network}
\pgfuseimage{overview}


\begin{scriptsize}Real time visuals: \url{http://bitcoinglobe.com/}\end{scriptsize}
\end{frame}

\section{Who}

\begin{frame}{Who Invented Bitcoin?}{Satoshi Nakamoto}

\begin{scriptsize}\begin{itemize}
\item Satoshi Nakamoto, in 2008 publishes a white paper, ``Bitcoin: a Peet-to-Peer Electronic Cash System'' via ``The Cryptography Mailing List''.
 \item In 2009--2011 he wrote a lot of posts (80000 words, the size of a novel) in flawless english with British colloquialisms (aside only the first post where he used American spellings).
 \item Satoshi is probably a pseudonym for a developer or a group, “vanished” from the web in April 2011 because he “moved to other things” 
 \item If he is not a group, he is a world class programmer, with deep knowledge of C++, economics, cryptography and peer-to-peer networking.
 \item His timestamps speculation are about either east-coast US with a fairly normal sleep schedule or western Europe with a \emph{coder} sleep schedule (probably not Japan)
\end{itemize}\end{scriptsize}

\challenge
\end{frame}

\section{Why}

\begin{frame}{Trusted third party}

\citazione{What is needed is an electronic payment system based on cryptographic proof
instead of trust, allowing any two willing parties to transact directly with each other 
without the need for a trusted third party.”}{Satoshi Nakamoto,
"Bitcoin: A Peer-to-Peer Electronic Cash System” -- October 31, 2008}

\begin{exampleblock}{}
Trustless does not mean that we do not need to trust \emph{anything}, but that we do not need to trust \emph{anyone}. 
\end{exampleblock}

\end{frame}

\section{How}

\begin{frame}{Consensus in a decentralized system}{The Byzantine Generals' problem}

\citazione{A group of generals of the Byzantine army camped with their troops around an 
enemy city. Communicating only by messenger, the generals must agree upon a 
common battle plan. However, one or more of them may be traitors who will try to 
confuse the others. The problem is to find an algorithm to ensure that the loyal 
generals will reach agreement.}{Marshall Pease, Robert Shosthak and Leslie Lamport, The Byzantine Generals Problem}

\end{frame}

\begin{frame}{Transactions}
  \pgfuseimage{transaction}
  
  \begin{minipage}{0.8 \textwidth}
   \citazione{\textbf{spending} is signing a transaction which transfers value from a previous transaction over to a new owner identified by a bitcoin address}{Andreas M. Antonopoulos -- Mastering Bitcoin -- O'Reilly 2014}
  \end{minipage}
\end{frame}

\begin{frame}{A Paper Wallet}{Vires in numeris}
 \pgfuseimage{wallet}
 \vfill
 \begin{tiny}
 \url{https://blockchain.info/address/1fbk5AYjA7wLdwbru2CunWEuToBu1USsX}
 \vfill
 \pgfuseimage{keytoaddr}
 \vfill
 Elliptic Curve Digital Signature Algorithm -- secp256k1 \\
 Hash = RIPEMD160(SHA256(pubkey)
\end{tiny}
\end{frame}

\section{The Blockchain}

\begin{frame}{Chain of Blocks}{A Distributed Ledger}
\begin{columns}
 \begin{column}{0.3 \textwidth}
  \pgfuseimage{chain}
 \end{column}
 \begin{column}{0.6 \textwidth}
 \begin{exampleblock}{\begin{scriptsize}Secure Hash Algorithm -- SHA256\end{scriptsize}}
  \begin{tiny}The proof of work used in Bitcoin takes advantage of the apparently random 
 nature of cryptographic hashes. A good cryptographic
 hash algorithm converts arbitrary data into a seemingly-random number.
 \end{tiny}
 \end{exampleblock}
  \pgfuseimage{merkle}
 \end{column}
 \hfill
\end{columns}
\end{frame}

\begin{frame}{Consensus via Proof of work}{Longest chain wins}
 \begin{columns}
  \begin{column}{0.5 \textwidth}
   \pgfuseimage{fullnode}
  \end{column}
  \begin{column}{0.4 \textwidth}
   \begin{exampleblock}{\begin{scriptsize}The ``Work'' is called ``mining'' \end{scriptsize}}
     \begin{scriptsize}
      \begin{enumerate}
       \item SHA256(SHA256(block header) + nonce) < \emph{target} ?
       \item The Bitcoin Network will rewards me (25 BTC)
      \end{enumerate}
      Difficulty (``inverse'' of target) will adapt to global hashrate every $\sim$ 2 weeks (2016 blocks)
     \end{scriptsize}
   \end{exampleblock}
  \end{column}
  \hfill
 \end{columns}
\end{frame}

\begin{frame}{Historycal Hashrate}{Logarithmic Scale}
 \pgfuseimage{hashrate}
\end{frame}

\section{Status}

\begin{frame}{Random Quotes}{Taking Breath}

\citazione{Bitcoin is a remarkable cryptographic achievement and the ability to create 
something that is not duplicable in the digital world has enormous value.}{Eric Schmidt (Google's former CEO)}

%\citazione{Every informed person needs to know about Bitcoin because it might be one of the world’s most important developments.}{Leon Louw, Nobel Peace prize nominee}

\citazione{Not having an internet strategy in 1995 is the equivalent of not having a bitcoin strategy now.}{
Moe Levin (Bitpay CEO)}

\citazione{By 2005 or so, it will become clear that the Internet's 
impact on the economy has been no greater than the fax machine's.}{Paul Robin Krugman
-- Nobel Memorial Prize in Economic Sciences (1998)}
\end{frame}


\begin{frame}{What is happening?}{Status of Venture Capitals}
  \pgfuseimage{vcmoney}
\end{frame}

\begin{frame}{Usage Metrics}{Latest quarter}
\pgfuseimage{metrics}
\begin{itemize}
 \item \url{http://coinmap.org/} 
\end{itemize}
\end{frame}

\section{Altcoins}

\begin{frame}{Crypto currencies}{640 currencies should be enough for everyone}
\pgfuseimage{cryptocap}
\end{frame}


\begin{frame}{Bitcoin is difficult}{Why FBK?}

 \begin{itemize}
  \item ICT is about data, communication and technologies
  \item Money is Data, Data is Money (Big Data, Secure computing)
  \item Cryptography is hard
  \item Peer-to-peer is hard
  \item Enterprenuers need Technology partners
 \end{itemize}
 


\end{frame}






\section{Final Notes}

\begin{frame}{What Next?}
 \begin{itemize}
  \item FBKcoin -- The ``Kessler''®
  \item PATcoin -- meal vouchers, \emph{glocal} social credits
  \item Trentino as the new ``Bitcoin Valley''
  \item Deep social and economic impact papers
  \item The Bitcoin Research Center
  \item \st{A new Research Unit}
 \end{itemize}
 
  \begin{block}{These slides}
  \url{http://goo.gl/BbzhTT} [github: mammadori]
  \end{block}

\end{frame}


\end{document}

